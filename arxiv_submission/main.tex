\documentclass[12pt]{article}
\usepackage{amsmath,amssymb,url}
\usepackage[utf8]{inputenc}
\usepackage{graphicx}

\title{Efficient Synchronization in Pulse-Coupled Entity Networks}
\author{Christopher Brown}
\date{October 2024}

\begin{document}

\maketitle

\begin{abstract}
We examine synchronization dynamics in networks of pulse-coupled computational entities and observe interesting scaling patterns. Our experiments show memory usage growing sub-linearly with entity count, with 1024 entities operating within 35.8MB of memory. We explore three architectural variations—cluster-based synchronization, domain-flexible entities, and compressed state representation—that maintain synchronization while demonstrating computational efficiency. These empirical observations may inform future work on distributed AI systems.
\end{abstract}

\section{Introduction}
Recent approaches to artificial intelligence scaling have predominantly focused on increasing model parameters, often accompanied by linear or super-linear computational costs. We investigate an alternative direction using synchronized networks of simple computational entities. While traditional distributed systems theory typically predicts linear memory scaling with component count, we observe patterns where carefully designed pulse-coupled networks can maintain synchronization while exhibiting sub-linear memory growth.

\section{Methodology}

\subsection{Entity Design}
Each computational entity implements phase synchronization dynamics inspired by Kuramoto oscillator models, combined with domain-specific behavioral patterns and local state vector maintenance. The base entity structure includes phase evolution and coupling mechanisms that enable network-wide synchronization.

\subsection{Architectural Variations}
We implement and compare three distinct architectural approaches:

\textbf{Cluster-based synchronization}: Entities are organized into clusters with representative nodes managing group coordination, reducing overall communication overhead while maintaining global synchronization.

\text
